\documentclass[10pt,twocolumn,letterpaper]{article}

\usepackage{cvpr}
\usepackage{times}
\usepackage{booktabs}
\usepackage{indentfirst}
\usepackage{epsfig}
\usepackage{float}
\usepackage{picinpar,graphicx}
\usepackage[breaklinks=true,bookmarks=false,colorlinks,
            linkcolor=red,
            anchorcolor=blue,
            citecolor=green,
            backref=page]{hyperref}

\cvprfinalcopy % *** Uncomment this line for the final submission
\def\cvprPaperID{****} % *** Enter the CVPR Paper ID here
\def\httilde{\mbox{\tt\raisebox{-.5ex}{\symbol{126}}}}


\begin{document}

%%%%%%%%% TITLE
\title{Rotation Averaging and Strong Duality}

\author{Wenjie Niu\\\\ June 30,2018}

\maketitle
%\thispagestyle{empty}

%%%A%%%%%% ABSTRACT
\begin{abstract}
The authors explore the role of duality principles
within the problem of rotation averaging, a fundamental
task in a wide range of computer vision applications. In its conventional form, rotation averaging is stated as a minimization over multiple rotation constraints. As these constraints are non-convex, this problem is generally considered challenging to solve globally.\par
\end{abstract}

\begin{figure}[!htp]
\begin{center}
   \includegraphics[width=1\linewidth]{Solutions.png}
\end{center}
   \caption{In many structure from motion pipelines, camera
orientations are estimated with rotation averaging followed
by recovery of camera centres (red) and 3D structure
(blue). Here are three solutions corresponding to different
local minima of the same rotation averaging problem.\cite{Eriksson_2018_CVPR}}
\label{fig:Solutions}
\end{figure}

%%%%%%%%% BODY TEXT
\section{Introduction}
Rotation averaging appears as a subproblem in many
important applications in computer vision, robotics, sensor
networks and related areas. Given a number of relative
rotation estimates between pairs of poses, the goal is to
compute absolute camera orientations with respect to some
common coordinate system. In computer vision, for instance,
non-sequential structure from motion systems such
as~\cite{Martinec2007Robust},\cite{Enqvist2011Non} rely on rotation averaging to initialize bundle
adjustment. The overall idea is to consider as much data as
possible in each step to avoid suboptimal reconstructions.
In the context of rotation averaging this amounts to using as
many camera pairs as possible.\par
Indeed, both $L_1$ and $L_2$ formulations of rotation averaging
can have local minima, see Fig.~\ref{fig:Solutions}. Wilson \emph{et al.}~\cite{Wilson_2016_wh} studied
local convexity of the problem and showed that instances
with large loosely connected graphs are hard to solve with
local, iterative optimization methods.\par

%-------------------------------------------------------------------------

{\small
\bibliographystyle{ieee}
\bibliography{RotationAveragingandStrongDuality}
}

\end{document}
