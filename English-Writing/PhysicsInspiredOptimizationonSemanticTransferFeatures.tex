\documentclass[10pt,twocolumn,letterpaper]{article}
\bibliographystyle{plain}
\usepackage{graphicx}
\usepackage{picinpar,graphicx}
\usepackage{cite}
\usepackage{booktabs}
\usepackage{float}
\usepackage{indentfirst}
\usepackage[pagebackref=true,breaklinks=true,letterpaper=true,colorlinks,bookmarks=false]{hyperref}

\begin{document}
\title{Physics Inspired Optimization on Semantic Transfer Features: An Alternative Method for Room Layout Estimation}
\author{Wenjie Niu}
\maketitle

%%%%%%%%% ABSTRACT
\begin{abstract}
   In this paper, we propose an alternative method to estimate
room layouts of cluttered indoor scenes. This method
enjoys the benefits of two novel techniques. The first one
is semantic transfer (ST), which is: (1) a formulation to
integrate the relationship between scene clutter and room
layout into convolutional neural networks; (2) an architecture
that can be end-to-end trained; (3) a practical strategy
to initialize weights for very deep networks under unbalanced
training data distribution. ST allows us to extract
highly robust features under various circumstances,
and in order to address the computation redundance hidden
in these features we develop a principled and efficient inference
scheme named physics inspired optimization (PIO).
PIO’s basic idea is to formulate some phenomena observed
in ST features into mechanics concepts. Evaluations on
public datasets LSUN and Hedau show that the proposed
method is more accurate than state-of-the-art methods.~\cite{Zhao_2017_CVPR}
\end{abstract}


\section{Introduction}
Given an input RGB image, a room layout estimation
algorithm should output all the wall-floor, wall-wall, and
wall-ceiling edges (depicted by Fig ~\ref{fig:overview}). This is a fundamental
indoor scene understanding task as it can provide
a strong prior for other tasks like depth recovery from a
single RGB image ~\cite{Eigen2014Depth}~\cite{Eigen_2014_CVPR} or indoor object pose estimation
~\cite{Song2014Sliding}~\cite{Gupta2015Aligning}~\cite{Song_2016_CVPR}. Besides, the room layout itself provides a highlevel
representation of an indoor scene for emerging applications
like intelligent robots and augmented reality. This
problem draws constant attention since the publication of
the seminal work~\cite{Hedau2010Recovering}, and there are two lines of followers:
(1) As the upper part of Fig~\ref{fig:overview}  shows, conventional methods
follow a proposing-ranking scheme. Typically, the proposing part consists of three sub-modules as edge detection,
vanishing point voting and ray sampling. With
hand-crafted features and structured inference techniques,
the ranking part outputs the best layout proposal, sometimes
along with a representation of the clutter.
(2) Recent methods~\cite{Mallya2015Learning}~\cite{Dasgupta2016DeLay}~\cite{Ren2016A} achieve dramatic performance
improvements via features produced by fully convolutional
networks (FCNs). ~\cite{Mallya2015Learning}~\cite{Ren2016A} still follow the traditional
proposing-ranking scheme. ~\cite{Dasgupta2016DeLay} is a proposal-free
solution in which all those steps about proposal generation
are eliminated. And instead of proposal ranking, in ~\cite{Dasgupta2016DeLay} inference
is achieved through an optimization module.\par

\begin{figure}[H]
\begin{center}
\includegraphics[scale=0.22]{Above.png}
\end{center}
   \caption{Above is the overview of conventional methods. Below is the overview of our method. Better viewed electronically.}
\label{fig:overview}
\end{figure}

\bibliography{PhysicsInspiredOptimizationonSemanticTransferFeatures}
\end{document}