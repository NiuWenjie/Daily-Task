\documentclass[10pt,twocolumn,letterpaper]{article}

\usepackage{cvpr}
\usepackage{times}
\usepackage{booktabs}
\usepackage{indentfirst}
\usepackage{epsfig}
\usepackage{picinpar,graphicx}
\usepackage[breaklinks=true,bookmarks=false,colorlinks,
            linkcolor=red,
            anchorcolor=blue,
            citecolor=green,
            backref=page]{hyperref}

\cvprfinalcopy % *** Uncomment this line for the final submission
\def\cvprPaperID{****} % *** Enter the CVPR Paper ID here
\def\httilde{\mbox{\tt\raisebox{-.5ex}{\symbol{126}}}}


\begin{document}

%%%%%%%%% TITLE
\title{Finding Tiny Faces in the Wild With Generative Adversarial Network}

\author{Wenjie Niu\\\\ June 10,2018}

\maketitle
%\thispagestyle{empty}

%%%%%%%%% ABSTRACT
\begin{abstract}
  \cite{Bai_2018_CVPR}Face detection techniques have been developed for
decades, and one of remaining open challenges is detecting
small faces in unconstrained conditions. The reason is
that tiny faces are often lacking detailed information and
blurring. In this paper, we proposed an algorithm to directly
generate a clear high-resolution face from a blurry small
one by adopting a generative adversarial network (GAN).
Toward this end, the basic GAN formulation achieves it by
super-resolving and refining sequentially (e.g. SR-GAN and
cycle-GAN). However, we design a novel network to address
the problem of super-resolving and refining jointly. We also
introduce new training losses to guide the generator network
to recover fine details and to promote the discriminator
network to distinguish real vs. fake and face vs. non-face simultaneously. Extensive experiments on the challenging
dataset WIDER FACE demonstrate the effectiveness of our
proposed method in restoring a clear high-resolution face
from a blurry small one, and show that the detection performance
outperforms other state-of-the-art methods.\par
\end{abstract}

%%%%%%%%% BODY TEXT
\section{Introduction}
Face detection is a fundamental and important problem
in computer vision, since it is usually a key step towards
many subsequent face-related applications, including
face parsing, face verification, face tagging and retrieval,
etc. Face detection has been widely studied over the past
few decades and numerous accurate and efficient methods
have been proposed for most constrained scenarios. Recent works focus on faces in uncontrolled settings, which is
much more challenging due to the significant variations in
scale, blur, pose, expressions and illumination. A thorough
survey on face detection methods can be found in~\cite{Zafeiriou2015A}.\par
Modern face detectors have achieved impressive results
on the large and medium faces, however, the performance
on small faces is far from satisfactory. The main difficulty
for small face (e.g. 10 $\times$ 10 pixels) detection is that small
faces lack sufficient detailed information to distinguish
them from the similar background, e.g. regions of partial
faces or hands. Another problem is that modern CNN-based
face detectors use the down-sampled convolutional (conv)
feature maps with stride 8, 16 or 32 to represent faces,
which lose most spatial information and are too coarse to
describe small faces. To detect small faces,~\cite{Xu2017Learning} directly
up-samples images using bi-linear operation and exhaustively
searches faces on the up-sampled images. However,
this method will increase the computation cost and the
inference time will increase significantly too. Moreover,
images are often zoomed in with a small upscaling factors
(2× at most) in~\cite{Xu2017Learning}, otherwise, artifacts will be generated.
Besides,~\cite{Bai2017Multi},~\cite{Jiang_2017_Det},~\cite{WanCZZW16},~\cite{ZhuZLS16} use the intermediate conv feature
maps to represent faces at specific scales, which keeps
the balance between the computation burden and the performance.
However, the shallow but fine-grained intermediate
conv feature maps lack discrimination, which causes many
false positive results. More importantly, these methods take
no care of other challenges, like blur and illumination.\par

\begin{figure}[!htb]
\begin{center}
   \includegraphics[width=1\linewidth]{ClearFaces.png}
\end{center}
   \caption{Some examples of the clear faces generated by our generator
network from the blurry ones. The top row shows the small
faces influenced by blur and illumination, and the bottom
row shows the clearer faces generated by our method. The lowresolution
images in the top row are re-sized for visualization.}
\label{fig:questions}
\end{figure}

\textbf{Influence of the GAN}. Table~\ref{tab:experiment} (the 1
st and the 5
th row)
shows the detection performance (AP) of the baseline detector
and our method on WIDER FACE validation set. Our
baseline detector is a multi-branch RPN face detector with
skip connection of feature maps, and please refer to~\cite{Bai2017Multi}
more detailed information . From Table~\ref{tab:experiment} we observe that
the performance of our detector outperforms the baseline
detector by a large margin (1.5\% in AP) on the Hard subset.
The reason is that the baseline detector performs the downsampling
operations (i.e. convolution with stride 2) on the
small faces. The small faces themselves contain limited information,
and the majority of the detailed information will
be lost after several convolutional operations. For example,
the input is a 16×16 face, and the result is 1×1 on the
C4 feature map and nothing is reserved on the C5 feature
map. Based on those limited features, it is normal to get the
poor detection performance. In contrast, our method first
learns a super-resolution image and then refines it, which
solves the problem that the original small blurry faces lack
detailed information and blurring simultaneously. Based on
the super-resolution images with fine details, the boosting
of the detection performance is inevitable.\par

\begin{table}[htbp] 
 \caption{Performance of the baseline model trained with and without
GAN, refinement network, adversarial loss and classification
loss on the WIDER FACE invalidation set.} 
 \label{tab:experiment}
 \begin{tabular}{cccc} 
  \toprule 
 Method & Easy & Medium & Hard \\
 \midrule 
Baseline\cite{Bai2017Multi} & 0.932 & 0.922 & 0.858 \\
w/o Refinement Network & 0.940 & 0.929 & 0.863  \\
w/o adv loss & 0.935 & 0.925 & 0.867  \\
w/o clc loss & 0.936 & 0.927 & 0.865  \\
Ours(Baseline+MES+adv+clc) & \textbf{0.944} & \textbf{0.933} & \textbf{0.873} \\
  \bottomrule 
 \end{tabular} 
\end{table}

\textbf{Influence of the refinement network}. From Table~\ref{tab:experiment} (the 2
nd and 5
th row), we see that the AP performance
increases by 1\% on the Hard subset by adding the refinement
sub-network to the generator network. Interestingly,
the performances of Easy and Medium subset also have an
improvement (0.4\%). We visualize the reconstructed faces
from the generator network and find that our refinement network
can reduce the influence of illumination and blur as
shown in Figure~\ref{fig:questions}. In some cases, the baseline detector fails
to detect the faces if those faces are heavily blurred or illuminated.
However, our method reduces influence of such
attributions and can find these faces successfully. Here, we
would like to note that our framework is not specific and
any off-the-shelf face detectors can be used as our baseline.
%-------------------------------------------------------------------------

{\small
\bibliographystyle{ieee}
\bibliography{FindingTinyFacesintheWildWithGenerativeAdversarialNetwork}
}

\end{document}