\documentclass{article}
\author{Wenjie Niu}
\title{Wikipedia has added page previews for easier browsing}
\begin{document}
\maketitle
Wikipedia added a useful new features earlier this week: page previews. It's said that it's one of the largest changes to desktop Wikipedia made in recent years and provides readers with a popup window providing a bit of additional context for the article behind the link.

Reading through any Wikipedia page can turn into a rabbit hole that can take you to places you never expected. I also think that exploration can be a fun, informative adventure, but it can also be a distraction, especially if the article you click on isn't actually useful.

The new page previews show an image and a couple of sentences that briefly describe the airticle when you hover your mouse over the link, providing a bit more context for you to decide whether or not you need to click on the link. Clicking on the pop up will take you to the article in question, and if you move the mouse away, it vanishes.

The new feature help the users with easier browsing. We do not have to click on the mouse and wait the website downloading. Just a briefly introduction, people are able to decide whether the content they need. 
\end{document}