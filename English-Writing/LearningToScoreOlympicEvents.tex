\documentclass{article} 
\author{Wenjie Niu}
\title{Learning To Score Olympic Events}
\usepackage{graphicx}
\usepackage{geometry}
\usepackage{float}
%\usepackage{multirow}%合并列时需要这个宏包
\usepackage{indentfirst}%使section后首段缩进
\begin{document} 
\twocolumn
\maketitle
I found an interesting paper named \emph{Learning To Score Olympic Events}. It mainly introduce a valid method to estimate sports action quality, which is complex to achieve because the process of assigning a "score" to the execution of an action contributes to difficulty as shown in Fig~\ref{fig:Gymnastic}. Unlike action recognition, which has millions of examples to learn from, the action quality datasets that are currently available are small – typically comprised of only a few hundred samples.\par
\begin{figure}[H]
\centering
 \includegraphics[scale=0.3]{Gymnastic.png} 
 \caption{Images from gymnastic vault, diving and figure skating datasets. For gymnastic vault, we illustrate the viewpoint variations; first row shows the take-off, second row shows the flight while the third row shows the landing; images from different samples are shown in different columns.}  
 \label{fig:Gymnastic}
 \end{figure}
This work presents three frameworks for evaluating Olympic sports which utilize spatiotemporal features learned using 3D convolutional neural networks (C3D) and perform score regression with
\begin{enumerate}
\item SVR.
\item LSTM.
\item LSTM followed by SVR.
\end{enumerate}

An efficient training mechanism for the limited data scenarios is presented for clip-based training with LSTM.While the SVR-based frameworks yield better results, LSTM-based frameworks are more natural for describing an action and can be used for improvement feedback.\par
\begin{table}
\centering
\begin{tabular}{|c|c|c|c|c|}
 \hline
 Datasets & \multicolumn{2}{|c|}{Driving} & Skating & Vault \\ \hline
 Samples & 1000/59 & 300/70 & 1000/70 & 120/56 \\ \hline
 C-S & 0.74 & 0.78 & 0.53 & 0.66 \\ \hline
 C-L(F) & 0.05 & 0.01 & - & -0.01 \\ \hline
 C-L(I) & 0.36 & 0.27 & - & 0.05 \\ \hline
 C-L-S(F) & 0.56 & 0.66 & - & 0.33 \\ \hline
 C-L-S(I) & 0.57 & 0.66 & - & 0.37 \\ \hline
\end{tabular}
\caption{Olympicscore prediction comparison with literature.(C = C3D, S = SVR, L = LSTM, F = Final, I =~Incremental).}
\label{Tab:experiment}
\end{table}
We evaluate the action quality assessment frameworks on three Olympic sports which are scored by judges,
\begin{enumerate}
\item figure skating
\item diving and
\item gymnastic vault.
\end{enumerate}

A full sum-mary of all results can be found in Table~\ref{Tab:experiment}.
\newpage
\bibliographystyle{plain} 
\bibliography{LearningToScoreOlympicEvents}
\nocite{*}%将没有引用的参考文献也表示出来
\end{document}