\documentclass[10pt,twocolumn,letterpaper]{article}

\usepackage{cvpr}
\usepackage{times}
\usepackage{booktabs}
\usepackage{indentfirst}
\usepackage{epsfig}
\usepackage{picinpar,graphicx}
\usepackage[breaklinks=true,bookmarks=false,colorlinks,
            linkcolor=red,
            anchorcolor=blue,
            citecolor=green,
            backref=page]{hyperref}

\cvprfinalcopy % *** Uncomment this line for the final submission
\def\cvprPaperID{****} % *** Enter the CVPR Paper ID here
\def\httilde{\mbox{\tt\raisebox{-.5ex}{\symbol{126}}}}


\begin{document}

%%%%%%%%% TITLE
\title{Learning Face Age Progression: A Pyramid Architecture of GANs}

\author{Wenjie Niu\\\\ June 12,2018}

\maketitle
%\thispagestyle{empty}

%%%%%%%%% ABSTRACT
\begin{abstract}
  \cite{Yang_2018_CVPR}The two underlying requirements of face age progression,
i.e. aging accuracy and identity permanence, are not
well studied in the literature. In this paper, we present a
novel generative adversarial network based approach. It
separately models the constraints for the intrinsic subjectspecific
characteristics and the age-specific facial changes
with respect to the elapsed time, ensuring that the generated
faces present desired aging effects while simultaneously
keeping personalized properties stable. Further, to
generate more lifelike facial details, high-level age-specific
features conveyed by the synthesized face are estimated by
a pyramidal adversarial discriminator at multiple scales,
which simulates the aging effects in a finer manner. The
proposed method is applicable to diverse face samples in
the presence of variations in pose, expression, makeup, etc.,
and remarkably vivid aging effects are achieved. Both visual
fidelity and quantitative evaluations show that the approach
advances the state-of-the-art.\par
\end{abstract}

%%%%%%%%% BODY TEXT
\section{Introduction}
Age progression is the process of aesthetically rendering
a given face image to present the effects of aging. It is often
used in entertainment industry and forensics, e.g., forecasting
facial appearances of young children when they grow up
or generating contemporary photos for missing individuals.\par
The intrinsic complexity of physical aging, the interferences
caused by other factors (e.g., PIE variations), and
shortage of labeled aging data collectively make face age
progression a rather difficult problem. The last few years
have witnessed significant efforts tackling this issue, where
aging accuracy and identity permanence are commonly
regarded as the two underlying premises of its success~\cite{Jinli2010A}\cite{Yang2016Face}\cite{Shu2015Personalized}\cite{Lanitis2009Evaluating}. The early attempts were mainly based on
the skin’s anatomical structure and they mechanically simulated
the profile growth and facial muscle changes w.r.t.
the elapsed time~\cite{Todd1980The}\cite{Ramanathan2008Modeling}. These methods provided the first insight into face aging synthesis. However, they generally
worked in a complex manner, making it difficult to
generalize. Data-driven approaches followed, where face
age progression was primarily carried out by applying the
prototype of aging details to test faces~\cite{Kemelmacher2014Illumination}\cite{Jinli2010A}, or by modeling
the dependency between longitudinal facial changes
and corresponding ages~\cite{Suo2012A}\cite{Wang2012Combining}\cite{Park2010Age}. Although obvious
signs of aging were synthesized well, their aging functions
usually could not formulate the complex aging mechanism
accurately enough,shown as Fig~\ref{fig:results}, limiting the diversity of aging patterns.\par
\begin{figure}[!htb]
\begin{center}
   \includegraphics[width=1\linewidth]{AgingSimulation.png}
\end{center}
   \caption{Demonstration of our aging simulation results (images
in the first column are input faces of two subjects).}
\label{fig:results}
\end{figure}

%-------------------------------------------------------------------------

{\small
\bibliographystyle{ieee}
\bibliography{LearningFaceAgeProgression}
}

\end{document}