\documentclass[10pt,twocolumn,letterpaper]{article}

\usepackage{cvpr}
\usepackage{times}
\usepackage{booktabs}
\usepackage{indentfirst}
\usepackage{epsfig}
\usepackage{float}
\usepackage{picinpar,graphicx}
\usepackage[breaklinks=true,bookmarks=false,colorlinks,
            linkcolor=red,
            anchorcolor=blue,
            citecolor=green,
            backref=page]{hyperref}

\cvprfinalcopy % *** Uncomment this line for the final submission
\def\cvprPaperID{****} % *** Enter the CVPR Paper ID here
\def\httilde{\mbox{\tt\raisebox{-.5ex}{\symbol{126}}}}


\begin{document}

%%%%%%%%% TITLE
\title{Cross-modal Deep Variational Hand Pose Estimation}

\author{Wenjie Niu\\\\ June 20,2018}

\maketitle
%\thispagestyle{empty}

%%%%%%%%% ABSTRACT
\begin{abstract}
The human hand moves in complex and highdimensional
ways, making estimation of 3D hand pose configurations
from images alone a challenging task. In this
work we propose a method to learn a statistical hand model
represented by a cross-modal trained latent space via a generative
deep neural network. We derive an objective function
from the variational lower bound of the VAE framework
and jointly optimize the resulting cross-modal KLdivergence
and the posterior reconstruction objective, naturally
admitting a training regime that leads to a coherent
latent space across multiple modalities such as RGB images,
2D keypoint detections or 3D hand configurations.
Additionally, it grants a straightforward way of using semisupervision.
This latent space can be directly used to estimate
3D hand poses from RGB images, outperforming the
state-of-the art in different settings. Furthermore, we show
that our proposed method can be used without changes
on depth images and performs comparably to specialized
methods. Finally, the model is fully generative and can
synthesize consistent pairs of hand configurations across
modalities. We evaluate our method on both RGB and depth
datasets and analyze the latent space qualitatively.\cite{Spurr_2018_CVPR}\par
\end{abstract}

%%%%%%%%% BODY TEXT
\section{Introduction}
Hands are of central importance to humans in manipulating
the physical world and in communicating with each
other. Recovering the spatial configuration of hands from
natural images therefore has many important applications
in AR/VR, robotics, rehabilitation and HCI. Much work exists
that tracks articulated hands in streams of depth images,
or that estimates hand pose~\cite{Oberweger2017DeepPrior},\cite{Oberweger2015Hands},\cite{Tang_2014_Latent},\cite{Wan2016Hand} from individual
depth frames. However, estimating the full 3D hand
pose from monocular RGB images only is a more challenging
task due to the manual dexterity, symmetries and selfsimilarities
of human hands as well as difficulties stemming
from occlusions, varying lighting conditions and lack of accurate
scale estimates. Compared to depth images the RGB
case is less well studied.\par
Recent work relying solely on RGB images~\cite{T_2017_L} proposes
a deep learning architecture that decomposes the task
into several substeps, demonstrating initial feasibility and
providing a public dataset for comparison. The proposed
architecture is specifically designed for the monocular case
and splits the task into hand and 2D keypoint detection followed
by a 2D-3D lifting step but incorporates no explicit
hand model. Our work is also concerned with the estimation
of 3D joint-angle configurations of human hands from RGB
images but learns a cross-modal, statistical hand model.
This is attained via learning of a latent representation that
embeds sample points from multiple data sources such as
2D keypoints, images and 3D hand poses. Samples from
this latent space can then be reconstructed by independent
decoders to produce consistent and physically plausible 2D
or 3D joint predictions and even RGB images.\par

\begin{figure}[!htp]
\begin{center}
   \includegraphics[width=1\linewidth]{LatentSpace.png}
\end{center}
   \caption{\textbf{Cross-modal latent space.} t-SNE visualization
of 500 input samples of different modalities in the latent
space. Embeddings of RGB images are shown in blue, embeddings
of 3D joint configurations in green. Hand poses
are decoded samples drawn from the latent space. Embedding
does not cluster by modality, showing that there is a
unified latent space. The posterior across different modalities
can be estimated by sampling from this manifold.}
\label{fig:Space}
\end{figure}

In this work we propose to learn a single, unified latent
space via an extension of the VAE framework. We provide a
derivation of the variational lower bound that permits training
of a single latent space using multiple modalities, where
similar input poses are embedded close to each other independent
of the input modality. Fig.~\ref{fig:Space} visualizes this learned
unified latent space for two modalities (RGB \& 3D). We
focus on RGB images and hence test the architecture on
different combinations of modalities where the goal is to
produce 3D hand poses as output. At the same time, the
VAE framework naturally allows to generate samples consistently
in any modality.\par

\begin{figure*}
\begin{center}
   \includegraphics[width=1\linewidth]{Architeture.png}
\end{center}
   \caption{\textbf{Schematic overview of our architecture.}Left: a cross-modal latent space z is learned by training pairs of encoder
and decoder q, p networks across multiple modalities (e.g., RGB images to 3D hand poses). Auxilliary encoder-decoder pairs
help in regularizing the latent space. Right: The approach allows to embed input samples of one set of modalities (here:
RGB, 3D) and to produce consistent and plausible posterior estimates in several different modalities (RGB, 2D and 3D).}
\label{fig:Architeture}
\end{figure*}

We deploy the VAE framework that admits cross-modal
training of such a hand pose latent space by using various
sources of data representation, even if stemming from different
data sets both in terms of input and output. Our crossmodal
training scheme, illustrated in Fig.~\ref{fig:Architeture}, learns to embed
hand pose data from different modalities and to reconstruct
them either in the same or in a different modality.\par
Fig.~\ref{fig:Architeture}, illustrates our proposed architecture for the case
of RGB based handpose estimation. In this setting we use
two encoders for RGB images and 3D keypoints respectively.
Furthermore, the architecture contains two decoders
for RGB images and 3D joint configurations.\par
%-------------------------------------------------------------------------

{\small
\bibliographystyle{ieee}
\bibliography{Cross-modalDeepVariationalHandPoseEstimation}
}

\end{document}