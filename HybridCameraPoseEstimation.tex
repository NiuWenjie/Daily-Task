\documentclass[10pt,twocolumn,letterpaper]{article}

\usepackage{cvpr}
\usepackage{times}
\usepackage{booktabs}
\usepackage{indentfirst}
\usepackage{epsfig}
\usepackage{float}
\usepackage{picinpar,graphicx}
\usepackage[breaklinks=true,bookmarks=false,colorlinks,
            linkcolor=red,
            anchorcolor=blue,
            citecolor=green,
            backref=page]{hyperref}

\cvprfinalcopy % *** Uncomment this line for the final submission
\def\cvprPaperID{****} % *** Enter the CVPR Paper ID here
\def\httilde{\mbox{\tt\raisebox{-.5ex}{\symbol{126}}}}


\begin{document}

%%%%%%%%% TITLE
\title{Hybrid Camera Pose Estimation}

\author{Wenjie Niu\\\\ July 2rd,2018}

\maketitle
%\thispagestyle{empty}

%%%A%%%%%% ABSTRACT
\begin{abstract}
This paper aims to solve the pose estimation problem of calibrated pinhole and generalized cameras w.r.t. a Structure-from-Motion (SfM) model by leveraging both 2D-3D correspondences as well as 2D-2D correspondences. While traditional approaches either focus on the use of 2D-3D matches, known as structure-based pose estimation or solely on 2D-2D matches (structure-less pose estimation).
\end{abstract}

\begin{figure}[!htp]
\begin{center}
   \includegraphics[width=1\linewidth]{Visualization.png}
\end{center}
   \caption{Visualization of 2D-2D matches (pink) and 2D-3D
matches (blue) used by one of our hybrid pose solvers. The query
camera is represented in red and SfM cameras in green.\cite{Camposeco_2018_CVPR}}
\label{fig:Visualization}
\end{figure}

%%%%%%%%% BODY TEXT
\section{Introduction}
Camera pose estimation, \emph{i.e.}, estimating the position and orientation of a given image, is a central step in 3D computer vision approaches such as SfM~\cite{Sameer_2009_ICCV}, Simultaneous Localization and Mapping (SLAM)~\cite{Davison2007MonoSLAM}, and visual localization~\cite{Camposeco2017Toroidal}. In addition, camera pose estimation plays an important role in applications such as selfdriving cars~\cite{HANE201714} and augmented reality~\cite{Middelberg2014Scalable}.\par
The availability of both structure-based and structureless camera pose estimation techniques leads to a set of interesting questions: Are they mutually exclusive, \emph{i.e.}, is one always preferable over the other, or is there value in using both 2D-3D and 2D-2D matches for pose estimation? Is it best to use ”pure” solvers, \emph{i.e.}, solvers that use either 2D-3D or 2D-2D correspondences, or do hybrid solvers (\emph{c.f.}  Fig.~\ref{fig:Visualization}) using both type of matches improve pose estimation performance? Should one decide prior to RANSAC which solver to use, or is it best to select solvers in a data-driven way during RANSAC-based pose estimation?\par

\section{Hybrid RANSAC for Pose Estimation}
RANSAC variant stops when at least one solver
$s$ has been chosen $K_s$ times, as this means that a good solution
for the current inlier ratios has been found with probability
$P$.

\section{Conclusions}
This paper have posed the question whether camera pose estimation can be improved by using both 2D-2D and 2D-3D matches. To answer this, we have developed a novel framework for camera pose estimation that jointly uses different minimal solvers within a new Hybrid RANSAC scheme.\par
\textbf{Acknowledgements.} This research was funded by Google.\par
%-------------------------------------------------------------------------

{\small
\bibliographystyle{ieee}
\bibliography{HybridCameraPoseEstimation}
}

\end{document}
