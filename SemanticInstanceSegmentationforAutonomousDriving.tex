\documentclass{article} 
\author{Wenjie Niu}
\title{Semantic Instance Segmentation for Autonomous Driving}
\usepackage{graphicx}
\usepackage{geometry}
\usepackage{float}
\usepackage{booktabs}%三线表宏包
\usepackage{indentfirst}%使section后首段缩进
\begin{document} 
\twocolumn
\maketitle
This is a summary of Semantic Instance Segmentation for Autonomous Driving, an article in \emph{CVPR},2017~\cite{Brabandere_2017_CVPR_Workshops}.
\begin{figure}[H]
\centering
 \includegraphics[scale=0.25]{Semantic.png} 
 \caption{Top: the network maps each pixel into feature space, in
which each object can be easily clustered with a fast clustering
approach. Bottom left: intra-cluster pull and inter-cluster push
forces. Bottom right: our method can handle complex occlusions,
useful for pick and place tasks.}  
 \label{fig:Semantic}
 \end{figure}
\section{Occlusion handling}
A key strength of our method is its ability to handle complex occlusions. Detect-and-segment approaches require an object’s segmentation mask to be unambiguously extracted from its bounding box. This assumption is problematic for certain tasks. Consider
a pick-and-place task where overlapping stick-like objects
need to be segmented as in fig~\ref{fig:Semantic}. When two sticks overlap like two crossed swords, their bounding boxes are highly overlapping. Given only a detection in the form of a bounding box, it is exceedingly hard to unambiguously extract a segmentation mask of the indicated object. In contrast to methods that rely on bounding boxes, our method treats the image holistically and can learn to reason about occlusions.\par
\section{Scene understanding for autonomous driving}
We test our loss function on the challenging Cityscapes dataset, a multi-class semantic instance segmentation benchmark. To cope with the multi-class problem, we apply our loss function independently on each semantic class so that instances from different classes are free to occupy the same feature space. The semantic segmentation masks are obtained
with the ResNet-38 network. The same architecture, pretrained on Cityscapes semantic segmentation, is also adopted for our instance segmentation network. We train the model on the 2975 training images, resized to 768x384 and use Adam with learning rate of 1e-4 on a NVIDIA Titan X. With our loss we achieve competitive results on the Cityscapes leaderboard, see table~\ref{Tab:experiment}.\par
\begin{table}
\centering
\caption{Segmentation results of best performing entries on the test set of the Cityscapes instance segmentation benchmark.}
\begin{tabular}{ccccc}
 \toprule
% \cmidrule{2-4}
               & AP   & AP0.5 & AP100m & AP50m \\
 \midrule
 InstanceCut   & 13.0 & 27.9  & 22.1   & 26.1  \\
         DWT   & 15.6 & 30.0  & 26.2   & 31.8  \\
 Shape-aware   & 17.4 & 36.7  & 29.3   & 34.0  \\
 Pixelwise DIN & 20.0 & 38.8  & 32.6   & 37.6  \\
          Ours & 17.5 & 35.9  & 27.9   & 31.0  \\ 
\bottomrule
\end{tabular}
\label{Tab:experiment}
\end{table}
\newpage
\bibliographystyle{plain} 
\bibliography{SemanticInstanceSegmentationforAutonomousDriving}
%\nocite{*}%将没有引用的参考文献也表示出来
\end{document}