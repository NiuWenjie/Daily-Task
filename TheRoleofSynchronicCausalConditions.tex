\documentclass[twocolumn]{article}
\bibliographystyle{IEEEtran}
\usepackage{indentfirst}
\usepackage{picinpar,graphicx}
\usepackage{cite}
\usepackage{float}
%\usepackage{amsmath}
%\usepackage{amssymb}
%\usepackage{fontspec}
\usepackage[pagebackref=true,colorlinks,linkcolor=red,citecolor=green]{hyperref}
%\DeclareMathOperator*{\argmax}{argmax}
\setlength{\parindent}{2em}
\author{Wenjie Niu}
\title{The Role of Synchronic Causal Conditions in Visual Knowledge Learning}
\begin{document}
	\maketitle
	\par
\section{Diachronic and Synchronic Causal Conditions}
\begin{figure}[H]
\centering
 \includegraphics[scale=0.25]{SynchronicCausalConditions.png} 
 \caption{Example used to illustrate the idea of diachronic and
synchronic causal conditions.}  
 \label{fig:Conditions}
 \end{figure}
\cite{Ho_2017_CVPR_Workshops}Some previous work has shown that it is essential to separate two basic kinds of causal conditions – the diachronic and synchronic causal conditions. We use a simple scenario in Figure~ref{}to illustrate this idea. \par
In Figure~\ref{fig:Conditions} it is shown that there are 3 objects, A, B, and C in an environment represented by the rectangle. Suppose we ignore the presence and influence of the rectangle and consider just the objects. \par
The idea of “enabling” causal
conditions has been investigated by Abelson in an earlier effort. If we think of it in terms of counterfactual function, then it is like “had it not been there, the diachronic cause would not have given rise to the effect.” Its counterfactual causal role is the same as that of the diachronic causal condition – “had the diachronic cause not been there, the effect would not have taken place.” In the following discussions we will not explicitly label DIA and SYN for the sake of succinctness but their respective
roles will be obvious.\par
\bibliography{TheRoleofSynchronicCausalConditions}
\end{document} 